% v2-acmtog-sample.tex, dated March 7 2012
% This is a sample file for ACM Transactions on Graphics
%
% Compilation using 'acmtog.cls' - version 1.2 (March 2012), Aptara Inc.
% (c) 2010 Association for Computing Machinery (ACM)
%
% Questions/Suggestions/Feedback should be addressed to => "acmtexsupport@aptaracorp.com".
% Users can also go through the FAQs available on the journal's submission webpage.
%
% Steps to compile: latex, bibtex, latex latex
%
% For tracking purposes => this is v1.2 - March 2012
\documentclass{acmtog} % V1.2
\usepackage[utf8]{inputenc} % Unicode funktioniert unter Windows, Linux und Mac
\usepackage[T1]{fontenc}
% \usepackage{hyperref} % Hyperref funktioniert leider nicht richtig mit bibitem
\usepackage{graphicx}
\usepackage{tikz}
\usepackage[scaled]{helvet}
\usepackage{rotating}
\usepackage[T1]{fontenc}
\usepackage{color}
\usepackage{cclicenses}

%\acmVolume{VV}
%\acmNumber{N}
%\acmYear{YYYY}
%\acmMonth{Month}
%\acmArticleNum{XXX}
%\acmdoi{10.1145/XXXXXXX.YYYYYYY}

\acmVolume{28}
\acmNumber{4}
\acmYear{2009}
\acmMonth{September}
\acmArticleNum{106}
\acmdoi{10.1145/1559755.1559763}

\begin{document}

\markboth{Shahin Imanverdiyev \& Senad Ličina}{Modelling and Conception for Security in Software}

% @shahin: this title is not very good, lets find a better one. our topic is "Modellierung und Entwurf für Security"
\title{Modelling and conception of software in aspect to security} % title

\author{
% \includegraphics[width=3.5em]{img/uhh}
\parbox{4.2em}{\includegraphics[width=3.5em]{img/uhh}}%
\parbox{\textwidth}{
Seminar: Software Architecture \\
Universität Hamburg \\
Shahin Imanverdiyev \& Senad Ličina
}
}

%\category{I.3.7}{Computer Graphics}{Three-Dimensional Graphics and Realism}[Animation]

%\terms{Experimentation, Human Factors}

%\keywords{Face animation, image-based modelling, iris animation, photorealism, physiologically-based modelling}

\maketitle

\begin{bottomstuff}
\today
% Senad Ličina $\bullet$ Seminar: Bio-Inspired Artificial Intelligence $\bullet$ Winter Term 2013/2014
\end{bottomstuff}

\begin{abstract}
%\textcolor{red}{\textbf{DRAFT:} This paper is still under construction!}
\begin{center}
ABSTRACT
\end{center}
% put abstract here.
put Abstract here.
\end{abstract}

\section{Introduction}

% security is important
Modern society and economy increasingly depend on digital systems.
As attacks against such systems can have devastating results, it is very important to secure such systems in a proper way.
Nowadays the Internet is a heavily used and very important communication medium.
An increasing number of devices are connected to the Internet, which is why it is important to transmit and store sensitive data securely.

% Developing secure systems is difficult
The correct development of secure software is difficult.
There have been numerous successful attacks abusing vulnerabilities of software systems in the past and it is to expect that people will try to abuse such flaws in the future as well.

% System development has to take security aspects into account.
The traditional approach for security assurance has been ``penetrate and patch'', where security is assured by attempting to break into a running system and exploiting well-known vulnerabilities.
Penetrate and patch happens too late in the development process and vulnerabilities will be available and possibly exploited  until they are recognized fixed.
This is why it is important to take security aspects into account in an early stage of the system development.

% Why we should scrap penetrate-and-patch - http://www.cigital.com/papers/download/compass-2.pdf

\section{Requirements}
\label{sec:requirements}
insert requirements here

\subsection{Key-Concepts}



Citation goes like this... \cite{SHRB11}
footnotes \footnote{this is a footnote.} are possible aswell.

Reference to Section \ref{sec:blank}.

blabla wichtig

intentionally left blank.

\subsection{Time-of-flight}

Look a figure (Figure \ref{fig:figure}).
\begin{figure}[ht]
\centerline{\includegraphics[width=4.5cm]{img/uhh}}
\caption{With Caption.}
    \label{fig:figure}
\end{figure}

\section{Conclusion}
\label{sec:conclusion}

Typical Conclusion.

\section{References}
\renewcommand{\section}[2]{}
\begin{raggedright}%schaltet Blocksatz ab, erzeugt ein stimmigeres Schriftbild im Literaturverzeichnis
\begin{thebibliography}{X}
    \bibitem[SHRB11]{SHRB11} Matthias Straka, Stefan Hauswiesner, Matthias Rüther, Horst Bischof: Skeletal Graph Based Human Pose Estimation in Real-Time. Graz University of Technology, Austria 2011.
\end{thebibliography}
\end{raggedright}

\end{document}
